\documentclass[a4paper]{spie}  %>>> use this instead for A4 paper

\renewcommand{\baselinestretch}{1.0} % Change to 1.65 for double spacing

\usepackage{amsmath,amsfonts,amssymb}
\usepackage{graphicx}
\usepackage[colorlinks=true, allcolors=blue]{hyperref}

\title{CI/CD practices with the TANGO-controls framework in the context of the Square Kilometre Array (SKA) telescope project}

\author[a]{Di Carlo M.}
\author[b]{Yilmaz U.}
\author[b]{Harding P.}
\author[b]{Bartolini M.}
\author[c]{Le Roux G.}
\author[a]{Dolci M.}

\affil[a]{INAF Osservatorio Astronomico d'Abruzzo, Teramo, Italy}
\affil[b]{SKA Organisation, Macclesfield, UK}
\affil[c]{SKA South Africa, SA}

\authorinfo{Further author information: (Send correspondence to Di Carlo M.)\\Di Carlo M.: E-mail: matteo.dicarlo@inaf.it\\  Dolci M.: E-mail: mauro.dolci@inaf.it\\ Harding P.: E-mail: P.Harding@skatelescope.org\\ Bartolini M.: E-mail: M.Bartolini@skatelescope.org\\ Yilmaz U.: E-mail: u.yilmaz@skatelescope.org}

% Option to view page numbers
\pagestyle{empty} % change to \pagestyle{plain} for page numbers
\setcounter{page}{301} % Set start page numbering at e.g. 301

\usepackage{xcolor}
\begin{document}
\maketitle

\begin{abstract}
The Square Kilometre Array (SKA) project is an international effort to build two radio interferometers in South Africa and Australia to form one Observatory monitored and controlled from the global headquarters (GHQ) based in the United Kingdom at Jodrell Bank. The project is now approaching the end of its design phase and gearing up for the beginning of formal construction. The period between the end of the design phase and the start of the construction phase, has been called bridging and, one of its main goals is to promote some CI/CD practices among the software development teams. CI/CD is an acronym that stands for continuous integration and continuous delivery and/or continuous deployment. Continuous integration (CI) is the practice of merging all developers local (working) copies into the mainline very often (many times per day). Continuous delivery is the approach of developing software in short cycles ensuring that it can be released anytime and continuous deployment is the approach of delivering the software into operational use frequently and automatically. The present paper analyses the decisions taken by the Systems Team (a specialized agile team devoted to developing and maintaining the tools that allows continuous practises) together with SKA architects in order to promote the CI/CD practices with the TANGO controls framework.
\end{abstract}

% Include a list of keywords after the abstract
\keywords{CI/CD, SKA, TANGO, Continuous Integration, Continuous Delivery, Systems Team, TANGO controls framework, Bridging, Software Development}

\section{Introduction}
\label{sec:intro}  % \label{} allows reference to this section

When creating releases for the end-users, every large software endevour faces the problem of integrating the different parts of the software solution and bringing them to the production environment where users work. The problem arises when many parts of the project are developed independently for a period of time and when merging them into the same branch, the process takes more than what was planned. In a classic Waterfall Software Development process this is usual, but the same also happens when following the classic Git Flow, also known as feature-based branching, which is when a branch is created for a particular feature. Considering, for example, one hundred developers working in the same repository each of them creating one or two branches. When merging it can easily lead conflicts and it becomes impossible, for a single developer, to solve all of them thus creating delay in publishing any release (in literature this is called "merge hell"). This problem becomes evident especially working with over a hundred repositories with different underyling technologies. Therefore, it is essential to develop a standard set of tools and guidelines to systematically manage and control different phases of the software development life cycle throughout the organisation.

In the Square Kilometre Array (SKA) project, The selected development process is SAFe Agile (Scaled Agile framework) that is basically incremental and iterative with a specialized team (known as the Systems Team) devoted to supporting the Continuous Integration, Continuous Deployment, test automation and quality.

\subsection{TANGO-controls Overview}
One of the most important decisions taken by the SKA project is the adoption of the TANGO-controls\cite{tango-controls} framework which is a middleware for connecting software processes together mainly based on the CORBA standard (Common Object Request Broker Architecture), for the purposes of controlling physical hardware elements. The standard defines how to exposes the procedures of an object within a software process with the RPC protocol (Remote Procedure Call).  The TANGO framework extends the definition of an object with the concept of a Device which represents a real or virtual device to control.  This exposes commands (that are procedures), and attributes (like the state) and allows both synchronous and asynchronous communication with events generated from the attributes (for instance a change in an attribute value can generate an event). Fig.~\ref{fig:tangodatamodel}  shows a module view of the framework.

\begin{figure}[!htb]
   \centering
   \includegraphics*[width=1\columnwidth]{SimplifiedDataModel}
   \caption{TANGO-Controls simplified data model.}
   \label{fig:tangodatamodel}
\end{figure}

\subsection{Continuous Integration (CI)}
CI refers to a set of development practices that requires developers to integrate code into a shared repository several times a day. Each check-in is then verified by an automated build, allowing teams to detect problems as early as possible with early feedbacks about the state of the integration.
According to Martin Fowler~\cite{CI}, there are a number of best practices to implement to reach CI:
\begin{itemize}
    \item Maintain a single source repository (for each component of the system) and try to minimize the use of branching, in favor of a single branch of the project currently under development.
    \item Automate the build (possibly build all in one command).
    \item Together with the build, it must also run tests so as to make the software self-testing (testing is crucial in CI because all the benefits of it come only if the test suite is good enough).
    \item Every commit should build on an integration machine: the more the developers commit the better it is (common practice is at least once per day). In fact, this reduces the number of potential conflicts and once a conflict is found, since the change is small, the fix is easier (as a consequence if a build fails then it must be fixed immediately).
    \item Keep the build fast so that a problem in integration can be found quickly.
    \item Multi-stage deployment: every software build must be tested in different environments (testing, staging and so on).
    \item Make it easy for anyone to get the latest executable version: all programmers should start the day by updating the project from the repository.
    \item Everyone can see what’s happening: a testing environment with the latest software should be running.
\end{itemize}

\subsection{Continuous Delivery and Continuous Deployment (CD)}
Continuous delivery~\cite{CD} refers to an extension of CI that corresponds to automating the delivery of new releases of software in a sustainable way. The release frequency can be decided according to the business requirements but the greatest benefit is reached by releasing as quickly as possible.
The deployment has to be predictable and sustainable, irrespective of whether it is a large-scale distributed system, a complex production environment, an embedded system, or an app. Therefore the code must be in a deployable state. Testing is one of the most important activities and it needs to cover enough of your codebase.
While it is often assumed that frequent deployment means lower levels of stability and reliability in the systems, this is not the reality and, in general, in software the golden rule is “if it hurts, do it more often, and bring the pain forward” (~\cite{CD}, page 26).

There are many patterns around deployment and, nowadays, all of them are related somehow to the DevOps culture. According to ~\cite{DevOps}, "DevOps is the outcome of applying the most trusted principles from the domain of physical manufacturing and leadership to the IT value stream. [...] The result is world-class quality, reliability, stability, and security at ever lower cost and effort; and accelerated flow and reliability throughout the technology value stream, including Product Management, Development, QA, IT Operations, and Infosec". Practically it corresponds to an increased collaboration between development (intended as requirements analysis, development and testing) and operations (intended as deployment, operations and maintenance). In the era of mainframes applications, it was common to have the two areas managed by different teams with the end result of having the development team with low (or zero) interest in the operations aspects (managed by a different team) and vice versa. Having a shared responsibility means that development teams share the problems of operations by working together in automating deployment operations and maintenance. It is also very important that teams are autonomous: they should be empowered to deploy a change to operations with no fear of failures. This is only possible by supplying necessary testing/staging platform and required infrastructure tools so that developers  can engage with the platforms.
Moreover, automation is one of the key elements in implementing a DevOps strategy, as it allows the teams to focus on what is valuable (code development, test result, etc. and not the deployment itself) and it reduces human errors.
The importance of those practices can be summarized in reducing risks of integration issues, of releasing new software and overall in having a better software product.
Continuous deployment goes one step further as every single commit (!) to the software that passes all the stages of the build and test pipeline is deployed into the production environment (preferably automatically).

\section{Containerisation} \label{SKA-docker}
The \textit{system engineering} development process has been adopted in the initial design phase of the SKA project in order to reduce the complexity by dividing the project into simpler and easier to resolve elements. For every element of the system, an initial architecture has been developed, which comprises the software modules needed which corresponds to a repository (each of them is a component of the system).

Since all components need to get deployed and tested together, the first decision taken is on how they need to be packaged.  A container is a standard unit of software that packages up code and all its dependencies so that the component runs quickly and reliably across different computing environments. A \textit{Docker container image} is a lightweight, standalone, executable package of software that includes everything needed to run an application: code (or more in general binary), runtime, system tools, system libraries and settings.

The final product will be a containerized application which will be running in a system for managing those kind of applications. In specific the selection made for SKA was \textit{Kubernetes (K8s)} for container orchestration~\cite{kubernetes} and \textit{Helm Charts}~\cite{helm} for declaring runtime dependencies for K8s applications. In K8s everything is a resource which may be a service, a volume or a simple pod which is the smallest deployable units of computing consisting of one (or many) container. The resources live in a cluster (a set of machines connected together) and share a predefined network (for service discovery),storage and other resources like computing power. On the other hand, helm is a tool for managing Kubernetes deployments with charts where a chart is a package of pre-configured Kubernetes resources.

The SKA repository \textit{SKA-docker}~\cite{SKA-docker} contains the definitions of a containerized TANGO environment and two Helm Charts: tango-base and archiver. The first one enables the installation of base services for a TANGO environment such as MariaDB service container, DatabaseDS service container, the TANGO test device and so on. The archiver enables the HDB++~\cite{hdb} application which is composed of a MariaDB service container, a configuration manager and an event subscriber. Tango-base and archiver provides the basic services and tools for the developers to easily develop their own tango device servers and applications while maintaining a consistent environment.

Every Helm Chart contains at a minimum the information concerning the version of the docker image and the pull policy (how to retrieve the image) for the orchestration. It also contains the needed information to correctly initialize the TANGO database (configuration of devices) and how it is exposed for other applications for discovery in the cluster.

Througout the SKA repositories, \textit{Makefiles} are selected as an abstraction and organisation layer to eliminate language specific scripts for building, testing and deployment and to promote ease of use in CI/CD. The use of a \textit{Makefile} in each project simplifies the work of containerisation and, overall, the automation of the code building, testing and packaging processes. In fact, with one single command, it is possible to compile the project, generate the docker image and test it by dynamically installing the related Helm Chart for that purpose.
The Makefile also allows to push the docker images and Helm Charts to the SKA artifact repository and, in general, it also enables the reusability of same build tool chain in different environments such as local development and CI/CD lifecyle.


% \subsubsection{TANGO-util library chart}
% A library Chart is a type of Helm Chart that defines chart primitives or definitions which can be % shared by Helm templates in other charts. In SKAMPI, many charts are a collections of TANGO device % servers so it is possible to harmonize their definition with a library. Fig.~\ref{fig:values_data_model} shows a data model diagram for the harmonized values file.

% \begin{figure}[!htb]
%    \centering
%    \includegraphics*[width=0.8\columnwidth]{values_data_model}
%    \caption{Data model for the values file}
%    \label{fig:values_data_model}
% \end{figure}

\section{Sub-charts Architecture}

\subsection{Introduction to Helm}
A chart can have one or more dependencies charts, called sub-charts. According to the Helm documentation:
\begin{itemize}
    \item a sub-chart is stand-alone (cannot depend on a parent chart),
    \item a sub-chart cannot access the values of its parent,
    \item a parent sub-chart can override values for its sub-charts and
    \item all charts (parent and sub-chart) can access the global values.
\end{itemize}

Let’s consider two charts, A and B where A depends on B. The file Chart.yaml for the chart A will specify the dependency and in the values file it is possible for chart A to override any value of chart B. Fig.~\ref{fig:a_parent_b} shows how to do it.

\begin{figure}[!htb]
   \centering
   \includegraphics*[width=0.8\columnwidth]{A_parent_B}
   \caption{Chart A parent of chart B}
   \label{fig:a_parent_b}
\end{figure}

It is also important to consider the operational aspects of using dependencies which state that when Helm installs/upgrades a chart, the Kubernetes objects from the chart and all its dependencies are
\begin{itemize}
    \item aggregated into a single set; then
    \item sorted by type followed by name; and then
    \item created/updated in that order.
\end{itemize}
This means that if chart A defines the following K8s resources:
\begin{itemize}
    \item namespace “A-Namespace”
    \item statefulset “A-StatefulSet”
    \item service “A-Service”
\end{itemize}
and chart B defines the following K8s resources:
\begin{itemize}
    \item namespace “B-Namespace”
    \item statefulset “B-ReplicaSet”
    \item service “B-Service”
\end{itemize}
Then the result of the helm install command for chart A will be:
\begin{itemize}
    \item A-Namespace
    \item B-Namespace
    \item A-Service
    \item B-Service
    \item B-ReplicaSet
    \item A-StatefulSet.
\end{itemize}

\subsection{Architecture}

The SKA MVP Product Integration, or SKAMPI~\cite{SKAMPI}, is both the set of software artefacts, and the corresponding repository and continuous integration facilities that allow for the development, testing, and integration of the SKA prototype software systems. It represents the main effort to integrate the components from the different SKA elements with each other, with the goal to provide first deployable versions of SKA software. A partial dependency diagram for the helm charts available within this project is represented in fig.~\ref{fig:skampi}.

\begin{figure}[!htb]
   \centering
   \includegraphics*[width=1\columnwidth]{simple_skampi}
   \caption{SKAMPI}
   \label{fig:skampi}
\end{figure}

Leaving out the details of the specific components of SKA, fig.~\ref{fig:skampi} shows that all charts depend on some shared charts, like tango-base (refer to section \ref{SKA-docker}) and, optionally, archiver (refer to section \ref{SKA-docker}) and webjive~\cite{webjive}.
At the moment, this is modelled in the repository SKAMPI where all charts are installed with helm templating instead of the normal installation. There are a number of disadvantages in this model specifically:

\begin{itemize}
    \item Common testing, one place for all testing with an unclear distinction the various types of tests
    \item Not easy to find logs
    \item Same K8s namespace for many deployments
    \item No versioning: charts are not versioned
\end{itemize}

Because of the above problems, a new architecture has been selected which enables a single level hierarchy for the shared charts (related to TANGO) and umbrella charts for charts composition (i.e. specific deployment or testing purpose). The rational can be summarized in the following items:
\begin{itemize}
    \item every chart can be deployed with its own tango eco-system,
    \item every chart can have tango-base, webjive and the archiver as dependencies.
\end{itemize}

Fig. ~\ref{fig:tmc_shared_charts} shows how a generic chart (in this case the SKA component called TMC) which includes the shared charts. Every dependency must have a common condition on it, so that it will be possible to disable the shared charts if they are included in the parent umbrella. For instance if there is the need (for testing purposes) to have the TMC and the OET charts together the result will be fig. ~\ref{fig:tmc_oet_umbrella}.

\begin{figure}[!htb]
   \centering
   \includegraphics*[width=0.5\columnwidth]{tmc_shared_charts}
   \caption{Chart TMC with shared charts}
   \label{fig:tmc_shared_charts}
\end{figure}

\begin{figure}[!htb]
   \centering
   \includegraphics*[width=0.8\columnwidth]{tmc_oet_umbrella}
   \caption{Umbrella chart with tmc and oet charts}
   \label{fig:tmc_oet_umbrella}
\end{figure}

Fig. ~\ref{fig:umbrella_skampi} is the result of the refactoring of fig. ~\ref{fig:skampi} with the new architecture.

\begin{figure}[!htb]
   \centering
   \includegraphics*[width=1\columnwidth]{umbrella_skampi}
   \caption{Umbrella chart for skampi: initial model refactored}
   \label{fig:umbrella_skampi}
\end{figure}

\section{Pipeline} \label{pipeline}

In order to bring everything together for a complete CI/CD toolchain, GitLab~\cite{gitlab} was selected. The data model for a generic SKA software is shown in figure~\ref{fig:pipelinedatamodel}.

\begin{figure}[!htb]
   \centering
   \includegraphics*[width=0.8\columnwidth]{dataEntity}
   \caption{Pipeline definition data model.}
   \label{fig:pipelinedatamodel}
\end{figure}

The entry point of the diagram is the Pipeline that is composed by many jobs (i.e. shell scripts).  This has been standardised for each project regardless of the artifact each project delivers so that the same standardised steps for code/configuration and helm charts are followed:
\begin{itemize}
    \item Linting, where code is analysed against a set (or multiple sets) of coding rules in order to check if it follows the best practices decided;
    \item Build, where code is compiled and a docker image is created;
    \item Test, where the compiled package (and docker image) are tested; tests are grouped into Fast / Medium / Slow / Very Slow categories.
    \item Publish, where the coding artifacts are published;
    \item Pages, where test results, documentations and logs are published (the name comes directly by the GitLab technology).
\end{itemize}

The pipeline is respected as the main hub of the software development in which code is built, tested, verified, published and integrated. The above steps are used in local development (as same shell scripts are available thanks to the \textit{Makefile} targets), merge workflow, QA, integration and releasing. Moreover, by having almost similar environment for different stages of software lifecyle, the differences between development and operational are eliminated.

Fig.~\ref{fig:cicdruntime} shows (without a specific formalism) the run-time behaviour of the selected technologies working together. At the center of the diagram there is a kubernetes cluster defined for every project in the SKA telescope group (and called \textit{syscore}). Outside the K8s cluster, there are the GitLab code repositories and Pages\cite{gitlab} (part of GitLab which represent a generic artifact used for storing pipeline artifacts such as test result), the Nexus Artifact repository\cite{nexus} to store packaged code artifacts, the ELK stack (Elasticsearch, Logstash and Kibana)\cite{elastcsearch} for logging, prometheus\cite{prometheus} for metric collection and ceph\cite{ceph} for a distribute storage solution. 
Inside the cluster, in an isolated K8s namespace, there are the GitLab runner related K8s resources which check, every 30 seconds, if there are pending pipelines triggered manually or pulled by the the resources.  If the runner finds a pipeline, it creates a K8s pod which executes all the jobs defined in the configuration file (and part of the git revision). Each created pod can (potentially) deploy a (umbrella) chart needed for the specific testing of the repository in an isolated namespace (i.e. an isolated environment). The deployment installed can then be tested and the result of the job will be reported to gitlab producing artifacts that will be stored in the correct artifacts repository. During any stage of the pipeline, jobs could also download necessary dependencies from the artifact repository. Depending on the type of the job, the pipeline is also used for deploying of the permanent running version of the SKAMPI or any other resources that are needed. \textit{syscore} kubernetes cluster is also equipped with monitoring solutions to examine the health and performance of the cluster and any resource that is deployed in it. Storage and logging solutions are also integrated to provide a consistent logging and distributed storage framework to the resources as needed. Finally, this architecture of creation temporal k8s resources for the pipeline steps (testing, building, packaging, etc.) ensures that necessary environment for the the jobs are always clean (not affected by the previously run pipelines).

\begin{figure}[!htb]
   \centering
   \includegraphics*[width=0.8\columnwidth]{cicdruntime-v2}
   \caption{CICD at run time}
   \label{fig:cicdruntime}
\end{figure}

\section{Testing}
The most important best practice for CI is testing so the main question now is how a generic component of SKA can be tested with the above architecture?
In SKA, testing has been split into two distinct types: pre-deployment and post-deployment tests. The deployment happens when a runner execute a job with an environment keyword. By doing so, the job is linked to the k8s cluster \textit{syscore}.
While the pre-deployment tests (namely unit tests) are all made without the real system to be online, the other ones (namely integration and system tests) needs more than one component to be up and running as the tests are mostly using other services and applications.
SKA is composed by many different modules, each of them with its own repository and different requirements for the components needed for its integration and system testing. For each of them, an umbrella chart has been introduced which enabled the specific component to be deployed together with its requirements.
In specific, to enable the GitLab pipeline to deploy and test the specific component each repository must:
\begin{itemize}
    \item contain at least one helm chart
    \item have an environment
    \item have a Makefile for K8s testing
\end{itemize}

The test job, introduced in section \ref{pipeline}, is composed by the following steps (all made with the help of a Makefile):
\begin{itemize}
    \item install: installs the chart (with the sub-charts needed) in the namespace specified in the environment
    \item wait: wait for every container to be running
    \item test:
    \begin{itemize}
        \item Create a container in the namespace specified in the environment
        \item Run pytests inside the above container
        \item Return the tests results
    \end{itemize}
    \item post test: delete all resources allocated for the tests
\end{itemize}

The artifacts are the output of the tests and it will have the report both in xml and json but also other information like the pytest output so that next steps (mostly packaging and releasing) in the pipeline could be run.

\section{Development Workflow}
There are two important assumption to understand the SKA development workflow: master branch shall always be stable and branches shall be short lived. With the term stable, it is meant that master branch always compiles and all automated tests run successfully. This also means that every time a master branch results in a condition of instability, reverting to a condition of stability shall have precedence over any other activity on the repository.
As a result, the selected development workflow for SKA is the following:
\begin{itemize}
    \item A developer takes a copy of the current code base on which to work
    \item Work is started on a new branch based on the story being implemented
    \item As the developer advances in the implementation commits are done on the local git repo.
    \item Unit tests are written and run in the development environment until successfully executed
    \item Once the tests pass the developer pushes the changes into a remote branch
    \item The CI server (GitLab)
    \begin{itemize}
        \item Checks out changes when they occur
        \item Runs static code analysis and provide feedback to the developer
        \item builds the system and runs unit and integration tests on the branch
        \item Provide feedback to the developer about status of the tests (fail or success)
        \item Provide feedback about coverage metrics
    \end{itemize}
    \item Once all tests execute successfully on the branch, the developer makes a pull request (\textit{i.e. Merge Request in GitLab terms}) for merging the changes into master.
    \item As part of the pull request the code is reviewed and approved by other developers.
    \item Code is merged into master branch
    \item Then, the CI server (GitLab)
    \begin{itemize}
        \item Runs the whole pipeline again including all the tests on the master branch
        \item Releases deployable artefacts for testing (reports, code analysis, etc.)
        \item Assigns a build label to the version of the code it just built (i.e. docker image version)
        \item Alerts the team if the build or tests fail which fixes the issue asap
        \item Publishes the successfully build artifacts to the artifact repository
    \end{itemize}
\end{itemize}

\section{Conclusion}
The majority of the decisions taken by the Systems Team follow the workflow as described by the Continuous Integration process outlined in the Martin Fowler’s paper and inspired by the state-of-the-art industry practices\cite{DevOps, CI, CD}. In particular:
\begin{itemize}
    \item For each component of the system, there is only one repository with minimal use of branching that are short lived;
    \item build, tests and publish of artifacts are automated with the use of few commands;
    \item Every commit triggers a build in a different machine (a container within the K8s cluster);
    \item Once the artifacts are built (docker images, helm charts, etc.), the repository SKAMPI will create automatically a new deployment of the system and more tests are done at that level (i.e. system tests);
    \item Having a common repository (Nexus and GitLab page) for the code artifacts and for the test results artifacts make it very easy to download the latest changes from every team and for each component to enable fast development;
    \item The integration environment is accessible for every developer and, in specific, is a specific namespace in a K8s cluster.
\end{itemize}

In addition, with the new sub-charts architecture, integration testing is done within the repositories of teams and brought in to the SKAMPI integration testing repository when a new version is created.  This enables developers to test not only their own work in isolation but also their work in conjunction with the developments contributed by other teams.

\acknowledgments % equivalent to \section*{ACKNOWLEDGMENTS}

This work has been supported by Italian Government (MEF - Ministero dell'Economia e delle Finanze, MIUR - Ministero dell'Istruzione, dell'Università e della Ricerca).

% References
\bibliography{report} % bibliography data in report.bib
\bibliographystyle{spiebib} % makes bibtex use spiebib.bst

\end{document}
